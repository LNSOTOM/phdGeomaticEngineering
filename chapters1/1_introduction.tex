\chapter{General Introduction}\label{cha:introduction}

\section{Background}
Global change phenomenon has been impacted by anthropogenic activities and disturbances with an increase of carbon dioxide (Co2) in the atmosphere, causing drastic climate change and highly affecting ecosystems health. Ecosystems productivity and cycles respond in complex dynamics to climate change and other global change drivers such as the increased atmospheric Co2, raised sea-surface temperatures, ocean acidification, water balance, precipitations, soil erosion and extreme events \citep{malhiClimateChangeEcosystems2020, dronovaRemoteSensingPhenology2022}. Understanding how these dynamics manifest on ecosystems are critical to assess the impacts of climate change and their resilience to it.

This study focuses on terrestrial ecosystems, where remote sensing plays an important role to monitor Earth’s land cover by vegetation and assess changes. Vegetation is a main component of ecosystems that connects soil, atmosphere and water or moisture and plays an important role in land surface energy exchange, biogeochemical cycle and hydrological or water cycle \citep{zhangFractionalVegetationCover2019, liangChapter12Fractional2020}.
Consequently, there is a need to derive magnitudes of ecosystems composition, structure, and function at a range of spatial and temporal scales to explain the drivers of change. 
There are existing remote sensing measurements that includes ground-based plot and transect-based filed observations that are very detailed, spaceborne satellite observations that cover a much larger extent and a higher frequency of data acquisition (the whole globe and Australian country), and airborne unoccupied aerial systems (UAS) (i.e.‘drones’) that provides similar detail of information or high spatial resolution, than ground-based data with wider extent cover and data frequency acquisition.

High resolution remotely sensed data are well used to develop highly precise maps of ecosystem composition, structure, derived the characterisation of functional diversity (or biophysical products). Repeated scans across time can be used to detect and assess ecosystem changes. Abiotic (such as climatic and edaphic) factors are well understood to be primary drivers of variation in natural ecosystems across spatial and temporal dimensions. The way in which biotic factors interact at different scales is less well understood. This project will use high spatial and spectral resolution remotely sensed datasets to map and detect 
ecosystem change. The utility of RGB, multispectral and hyperspectral systems, as well as 3D datasets derived from both photogrammetry, and Light Detection And Ranging (LiDAR) will be investigated to identify 
vegetation cover, develop possible spectral libraries that contribute to essential biodiversity variables (EBV) and essential climate variables (ECV), and fil the scale gap between field-based measurement and satellite monitoring. The use UAS will facilitate the collection of ultra-high spatial resolution datasets. Machine learning techniques such as, deep learning algorithms will be used to develop models that can classify vegetation based on its’ spectral and structural datasets. Model interpretation and analysis by simulation will identify primary drivers of ecosystem composition, structure, develop monitoring protocols, and provide opportunities for management interventions to improve ecosystem function.

This research will further scientifically be understanding of ecosystem composition and resilience by developing remote sensing methodologies and algorithms to assess changes terrestrial ecosystems across a variety of Australian environments. The project will utilise machine learning techniques to perform classification to assist ecologists to map and understand the drivers of ecosystem change. As part of this research, the utility of RGB, multispectral, LiDAR and hyperspectral sensors will be compared. The project will also develop monitoring protocols that will enable efficient change detection and focused management interventions.


%%%%%%%%%%%%%%%%%%%%%%%%%%%%%%%%%%%%%%%%%%%%%%%%%%%%%%
\section{Motivation}
UAS data provide a unique opportunity to rapidly assess the response of ecosystems under 
environmental perturbation derived from abiotic factors (such as climate, soil type or other factors) by providing new insights into ecosystem structure and composition such as detecting vegetation stress across landscapes through these ultrahigh-resolution observations. These UAS observations can help to connect field plot surveys to coarser spatial scale satellite observation leading to improved insights and a better understanding in the satellite signal \citep{lucieer2014Using, melvilleUltrahighSpatialResolution2019}. UAS remote sensing has the potential to provide suitable datasets for calibration and validation of national-scale satellite vegetation products \citep{fiskComparisonHyperspectralTraditional2019}.

In order to recognise spatial patterns in ecosystem  composition and structure, ultra-high resolution remote sensing data such as that collected by UAS is required. The use of UAS compared to airborne and satellite platforms provides more details due to a higher spatial resolution, allowing higher details of the object detected. On the other hand, UAS compared to field-based observations enables a faster data collection, and at lower cost by providing a more flexible temporal resolution.
The rapid rate of change global climate means that there is an ever-increasing need for highly precise maps of ecosystem composition, structure, and function.


%%%%%%%%%%%%%%%%%%%%%%%%%%%%%%%%%%%%%%%%%%%%%%%%%%%%%%
\section{Problem statement}
UAS are increasingly used for ecosystem monitoring. There is a distinct need to get ecologically meaningful information from ultrahigh-resolution drone data by identifying the best approaches for image processing from raw data and, or analysis-ready data to high-level vegetation products. 

Deriving vegetation cover, structure and function from UAS data still in development. Methods for the creation of image derivates such as vegetation species richness, diversity, and abundance have not been universally defined. The use of machine learning algorithms provides new opportunities for feature extraction and classification from data with a high degree of spectral and structural complexity \citep{ghamisi2017Advanced}. The optimisation of model parameters specific to the characteristics of the study environment remains an issue within the remote sensing community. This project will aim to define optimal protocols for the collection, processing, and analysis of UAS data to derived vegetation products.  
 


%%%%%%%%%%%%%%%%%%%%%%%%%%%%%%%%%%%%%%%%%%%%%%%%%%%%%%
\section{Research Aims \& Objectives}
The aim of this project is to develop and apply novel image and 3D data analysis techniques to map vegetation composition, structure, and function from ultrahigh-resolution UAS data. In doing so, this project will provide new insights into fine-scale ecological patterns, processes, and dynamics thereby improving our ability to quantify ecosystem change and linking field observations to satellite observations through the use of UAS as a scaling tool.


%%%%%%%%%%%%%%%%%%%%%%%%%%%%%%%%%%%%%%%%%%%%%%%%%%%%%%
\section{Thesis Structure}
This thesis is divided into five/six  chapters.
Chapter one outlines key literature relevant to this research, highlights gaps in the literature, provides a study context and outlines the research aims and motivations behind the thesis. 

\newpage