\chapter{Introduction}\label{cha:introduction}

\section{Context}
Global change phenomenon has been impacted by anthropogenic activities and disturbances with an
increase of carbon dioxide (Co2) in the atmosphere, causing drastic climate change and highly
affecting ecosystems health. Ecosystems productivity and cycles respond in complex dynamics to
climate change and other global change drivers such as the increased atmospheric Co2, raised sea-
surface temperatures, ocean acidification, water balance, precipitations, soil erosion and extreme
events (\citealp{malhiClimateChangeEcosystems2020}). Understanding how these dynamics manifest
on ecosystems are critical to assess the impacts of climate change and their resilience to it.

This study focuses on terrestrial ecosystems, where remote sensing (RS) plays an important role to
monitor Earth’s land cover by vegetation and assess changes. Vegetation is a main component of
ecosystems that connects soil, atmosphere and water or moisture and plays an important role in land
surface energy exchange, biogeochemical cycle and hydrological or water cycle (Zhang, 2019, Liang, 2020).
Consequently, there is a need to derive magnitudes of ecosystems composition, structure, and function at
a range of spatial and temporal scales to explain the drivers of change. There are existing RS measurements
that includes ground-based plot and transect-based filed observations that are very detailed, spaceborne
satellite observations that cover a much larger extent and a higher frequency of data acquisition (the whole
globe and Australian country), and airborne unoccupied aerial systems (UAS) (i.e.‘drones’) that provides
similar detail of information or high spatial resolution, than ground-based data with wider extent cover
and data frequency acquisition.

High resolution remotely sensed data are well used to develop highly precise maps of ecosystem composition, 
structure, derived the characterisation of functional diversity (or biophysical products). Repeated scans 
across time can be used to detect and assess ecosystem changes. Abiotic (such as climatic and edaphic) 
factors are well understood to be primary drivers of variation in natural ecosystems across spatial and 
temporal dimensions. The way in which biotic factors interact at different scales is less well understood. 
This project will use high spatial and spectral resolution remotely sensed datasets to map and detect 
ecosystem change. The utility of RGB, multispectral and hyperspectral systems, as well as 3D datasets 
derived from both photogrammetry, and Light Detection And Ranging (LiDAR) will be investigated to identify 
vegetation cover, develop possible spectral libraries that contribute to essential biodiversity variables 
(EBV) and essential climate variables (ECV), and fil the scale gap between field-based measurement and 
satellite monitoring. The use UAS will facilitate the collection of ultra-high spatial resolution datasets. 
Machine learning techniques such as, deep learning algorithms will be used to develop models that can 
classify vegetation based on its’ spectral and structural datasets. Model interpretation and analysis by 
simulation will identify primary drivers of ecosystem composition, structure, develop monitoring protocols, 
and provide opportunities for management interventions to improve ecosystem function.

This research will further scientifically be understanding of ecosystem composition and resilience by 
developing remote sensing methodologies and algorithms to assess changes terrestrial ecosystems across a 
variety of Australian environments. The project will utilise machine learning techniques to perform 
classification to assist ecologists to map and understand the drivers of ecosystem change. As part of this 
research, the utility of RGB, multispectral, LiDAR and hyperspectral sensors will be compared. The project
will also develop monitoring protocols that will enable efficient change detection
and focused management interventions.

