\documentclass{thesis}
%%%%%%%%%%%%%%%%%%%%%%
\title{Improved monitoring of ecosystem resilience using high-resolution remote sensing}
\author{Laura Natali Sotomayor, \\
MITS (Hons)}
\degree{
	Submitted in fulfilment of the requirements for the degree of \\
Doctor of Philosophy in Geomatic Engineering\\
\vspace{1em}
School of Geography, Planning and Spatial Science}
%%%%%%%%%%%%%%%%%%%%%%
\begin{document}

\maketitle
 
\section*{Abstract}

% As technologies and computer storage capacity develops, there are ever increasing amounts of data available for us to use.  Despite this increase in available data, historically there have been limited benefits from it because of the issues involved in analysis of such large quantities because traditional statistical techniques are slow to process such large volumes. The recent development of machine learning algorithms and cloud technologies presents an unique opportunity to process these volumes of data, and have the potential to provide much greater accuracy and model predictive power in output results. 

% In this study, machine learning algorithms (MLAs) and cloud base systems have been used to assess the key drivers of forest productivity in radiata pine plantations in Tasmania. Data on 23 predictor variables known to impact forest productivity were collated, including climate, topographic and edaphic variables, and used these to test the accuracy of 5 different MLAs - Linear regression (LM), Polynomial regression (Poly\_reg), Decision Trees Regression (DTree), Random Forests (RF) and Gradient Boosted Regression Trees (GBDT). It was found that the best model was GBDT, which had predictive power of $>$ 0.845 for unseen data, although the other methods also proved successful with predictive powers of 0.85 (RF) and 0.703 (DTree) respectively. In these three most accurate models, rainfall was the most important factor driving forest productivity, followed by the topographic position index (TPI), geology composition (SYMBOL) and aspect landscape attribute.

% The outputs of models varied from model to model, and it was proven that adding a certain amount of complexity improves the quality of the model. This complexity allows a better understanding of the relationship between features and the interactions with the target variable. Tuning the model complexity is key to improving the data fitting process resulting in more accurate models.

% This work is important in demonstrating the usefullness of MLAs and cloud based systems to improve the accuracy and the efficiency of assessments and predictions using big data.

\clearpage

\section*{Acknowledgements}

\ifdrafting\else
\textbf{Dr. Arko Lucieer} (Head of School, Geography, Planning, and Spatial Sciences, University of Tasmania), 
\textbf{Dr. Bethany Melville} (Lecturer in Spatial Sciences) 
% The authors would like to thank Dr. Matthew J. Cracknell for assisting with field machine learning process, understanding of data processing, transformation and reprojection. Besides, my father \textbf{Jesus Sotomayor} (Petrophysicist Engineer/Geoscientist) has been assessing the machine learning process, mapping and geological interpretations and which comments have improved the clarity of this ICT Honours Master Thesis. Dr Muhammed was a great mentor for the shape of the document and iterative feedback. Dr. Musk gave me feedback and understanding of features that can affect the forest productivity, and a good understanding of LiDAR technology. Besides, my mum \textbf{Patricia Ramirez} and \textbf{Robert Darwall} have been a great support for having all set up at home during this Covid time. In addition, my partner \textbf{Claire Butler} (Marine scientist) for all the support for keeping my light bright and meaningful comments to improve this Thesis.

\vspace{5mm}
\begin{quotation}
% \textit{"Uncertainty about future events is the reason for forecasting. Forecast or prediction only reduce uncertainty but it does not eliminate the uncertainty phenomena." \citep{krzysztofowicz2001case}}
\end{quotation}

\begin{quotation}
% \textit{"We are drowning in information and starving for knowledge"-Rutherford D. Roger \citep{hastie2009elements}}
\end{quotation}

\contentspage
%%%%%%%%%%%%%%%%%%%%%%
\pagestyle{header-footer}
\part{Introduction}
\chapter{Introduction}\label{cha:introduction}

\section{Context}
Global change phenomenon has been impacted by anthropogenic activities and disturbances with an
increase of carbon dioxide (Co2) in the atmosphere, causing drastic climate change and highly
affecting ecosystems health. Ecosystems productivity and cycles respond in complex dynamics to
climate change and other global change drivers such as the increased atmospheric Co2, raised sea-
surface temperatures, ocean acidification, water balance, precipitations, soil erosion and extreme
events (\citealp{malhiClimateChangeEcosystems2020}). Understanding how these dynamics manifest
on ecosystems are critical to assess the impacts of climate change and their resilience to it.

This study focuses on terrestrial ecosystems, where remote sensing (RS) plays an important role to
monitor Earth’s land cover by vegetation and assess changes. Vegetation is a main component of
ecosystems that connects soil, atmosphere and water or moisture and plays an important role in land
surface energy exchange, biogeochemical cycle and hydrological or water cycle (Zhang, 2019, Liang, 2020).
Consequently, there is a need to derive magnitudes of ecosystems composition, structure, and function at
a range of spatial and temporal scales to explain the drivers of change. There are existing RS measurements
that includes ground-based plot and transect-based filed observations that are very detailed, spaceborne
satellite observations that cover a much larger extent and a higher frequency of data acquisition (the whole
globe and Australian country), and airborne unoccupied aerial systems (UAS) (i.e.‘drones’) that provides
similar detail of information or high spatial resolution, than ground-based data with wider extent cover
and data frequency acquisition.

High resolution remotely sensed data are well used to develop highly precise maps of ecosystem composition, 
structure, derived the characterisation of functional diversity (or biophysical products). Repeated scans 
across time can be used to detect and assess ecosystem changes. Abiotic (such as climatic and edaphic) 
factors are well understood to be primary drivers of variation in natural ecosystems across spatial and 
temporal dimensions. The way in which biotic factors interact at different scales is less well understood. 
This project will use high spatial and spectral resolution remotely sensed datasets to map and detect 
ecosystem change. The utility of RGB, multispectral and hyperspectral systems, as well as 3D datasets 
derived from both photogrammetry, and Light Detection And Ranging (LiDAR) will be investigated to identify 
vegetation cover, develop possible spectral libraries that contribute to essential biodiversity variables 
(EBV) and essential climate variables (ECV), and fil the scale gap between field-based measurement and 
satellite monitoring. The use UAS will facilitate the collection of ultra-high spatial resolution datasets. 
Machine learning techniques such as, deep learning algorithms will be used to develop models that can 
classify vegetation based on its’ spectral and structural datasets. Model interpretation and analysis by 
simulation will identify primary drivers of ecosystem composition, structure, develop monitoring protocols, 
and provide opportunities for management interventions to improve ecosystem function.

This research will further scientifically be understanding of ecosystem composition and resilience by 
developing remote sensing methodologies and algorithms to assess changes terrestrial ecosystems across a 
variety of Australian environments. The project will utilise machine learning techniques to perform 
classification to assist ecologists to map and understand the drivers of ecosystem change. As part of this 
research, the utility of RGB, multispectral, LiDAR and hyperspectral sensors will be compared. The project
will also develop monitoring protocols that will enable efficient change detection
and focused management interventions.


\chapter{Literature Review}\label{cha:litReview}
\input{chapters/methodology.tex}
\input{chapters/analysis.tex}
\input{chapters/discussion.tex}
\chapter{Conclusion}\label{cha:conclusion}
\setcounter{chapter}{0}
%%%%%%%%%%%%%%%%%%%%%%

% \part{History of Lua\TeX}
\pagestyle{header-footer}
\part{Deriving fractional cover from drone multispectral and lidar data}
\chapter{Introduction}\label{cha:introduction2}

\section{Context}

% \chapter{References}\label{cha:references}

% \section{References}\label{cha:ref1}
% \bibliographystyle{ieeetr}
% \bibliography{citations_chap1}

% \nocitef{*}
% \bibliographystylef{apalike}
% \bibliography{citations_chap1}

% \putbib[]
\nociteI{*}
\bibliographyI{citationschap1}
\bibliographystyleI{kluwer}


\input{chapters1/methodology.tex}
\input{chapters1/analysis.tex}
\input{chapters1/discussion.tex}
\chapter{Conclusion}\label{cha:conclusion}
\setcounter{chapter}{0}

\part{Structure properties derived from drone lidar data}
\chapter{Introduction}\label{cha:introduction2}

\section{Context}

% \chapter{References}\label{cha:references}

% \section{References}\label{cha:ref1}
% \bibliographySecond{citations}
% \bibliographystyle{ieeetr}


% \bibliography{chapters2/citationsfinal}
% \bibliographystyle{kluwer}

\nociteII{*}
\bibliographyII{citationschap2}
\bibliographystyleII{kluwer}

% \nocites{*}
% \bibliographystyles{apalike}
% \bibliographys{citations_final}
\input{chapters2/methodology.tex}
\input{chapters2/analysis.tex}
\input{chapters2/discussion.tex}
\chapter{Conclusion}\label{cha:conclusion}
%%%%%%%%%%%%%%%%%%%%%%
\clearpage


\bibliography{citations.bib}\label{app:bibTex}

\appendix

\vspace{-1em}

%%%%%%%%%%%%%%%%%%%%%%
\section{Supplementary Information}\label{app:dataset}
% The information contained in this appendix concerns some extra analysis related to each site that could help to understand the relationship between the target variable and the features.

\subsection{Edaphic variables}
\textbf{Geology structure by site}

\begin{figure}[h!]
	\centering
	% \includegraphics[width=\textwidth]{assets/geolStrucBySite.png}
	% \caption{\label{fig:soilClass}Geology composition at a Site in percentages.}
\end{figure}

\textbf{Bins}

\begin{figure}[h!]
	\centering
	% \includegraphics[width=\textwidth]{assets/geolStrucBySite_Quantiles.png}
	% \caption{\label{fig:soilClass}Geology composition at a Site in quantiles by productivity.}
\end{figure}

\newpage
\textbf{Soil classification by site}

\begin{figure}[h!]
	\centering
	% \includegraphics[width=\textwidth]{assets/soilClassBySite.png}
	% \caption{\label{fig:soilClass}Soil classification at a Site in percentages.}
\end{figure}

\textbf{Bins}

\begin{figure}[h!]
	\centering
	% \includegraphics[width=\textwidth]{assets/soilClassBySite_Quantiles.png}
	% \caption{\label{fig:soilClass}Soil classification at a Site in quantiles by productivity.}
\end{figure}


%%%%%%%%%%%%%%%%%%%%%%%%%%%%%%%%%%%%%%%%%%%%%%%%%%%%%%
\clearpage
\section{Project Scripts}\label{app:scripts}

The project scripts are available online at:
\newline
\url{https://github.com/LNSOTOM/phdGeomaticEngineering}. 
\newline
The scripts are written in \textbf{Python 3.8}.


\end{document}
